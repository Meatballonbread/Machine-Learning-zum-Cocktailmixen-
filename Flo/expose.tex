\documentclass[12pt, a4paper]{report}
\usepackage[ngerman]{babel}
\usepackage[T1]{fontenc}
\usepackage[utf8]{inputenc}
\usepackage{geometry}
\usepackage{graphicx}
\usepackage{amsmath}
\usepackage{amssymb}
\usepackage[numbered]{bookmark}
\usepackage{setspace}
\usepackage{hyperref}
\onehalfspacing
\geometry{left=3cm, right=2.5cm, top=2.5cm, bottom=2.5cm}

\begin{document}

\begin{titlepage}
    \centering
    \vspace*{3cm}
    {\huge \textbf{Machine Learning zum Cocktailmixen} \par}
    \vspace{2cm}
    {\large Florian Grassl \par}
    \vspace{1cm}
    Fachsemester: 5 \par
    \vspace{1cm}
    Betreuer: Moritz Fuchsloch, Kim Glumann \par
    Institut für Strahlwerkzeuge (IFSW) \par
    Universität Stuttgart \par
    \vfill
    Stuttgart, \today
\end{titlepage}

\tableofcontents
\clearpage

\chapter{Einleitung}

\section{Motivation}
Seit Veröffentlichung von GPT-4 im November 2022 ist das öffentliche Interesse an Künstlicher Intelligenz (KI) und Machine Learning (ML) enorm gestiegen. Cocktails bieten ein ideales Anwendungsszenario für ML, da sie über einen großen Parameterraum verfügen und die menschliche Geschmackswahrnehmung sehr komplex und nicht-linear ist. Diese Arbeit demonstriert, wie ML eingesetzt werden kann, um automatisch Cocktails zu mischen, indem komplexe Geschmacksprofile analysiert und in individuelle Rezepturen übersetzt werden.

\section{Stand der Technik}
Im Bereich des maschinellen Lernens dominieren aktuell Transformer-Modelle aufgrund ihrer Fähigkeit, komplexe Zusammenhänge in umfangreichen Datensätzen effektiv zu erfassen \cite{vaswani2017attention}. Neben Transformern werden auch andere Modelle wie CNNs (für Bilder), RNNs (für sequentielle Daten) und GANs (zur Daten-Generierung) genutzt \cite{goodfellow2014generative}. Für diese Arbeit wurde jedoch bewusst ein Autoencoder gewählt, da dieser robust bei kleinen Datensätzen ist und besonders gut zur Rekonstruktion latenter Muster geeignet ist \cite{goodfellow2016deep}. Neuere Trends beinhalten hybride Modelle und Reinforcement Learning, um Nutzerfeedback einzubeziehen \cite{sutton2018reinforcement}.

\chapter{Hauptteil}
\section{Methodik}

\subsection{Datenvorverarbeitung}
Zu Beginn wurde eine JSON-basierte Datenbank angelegt, die alle relevanten Zutaten
sowie ihre zugehörigen Geschmacksprofile (süß, sauer, bitter, fruchtig, würzig) enthält.
Diese Datenbank dient als Grundlage für sämtliche Analysen und Optimierungen. Um
eine effiziente Datenverarbeitung zu ermöglichen, wurde der JSON-Datensatz zunächst
mithilfe von Python-Bibliotheken wie JSON und pandas eingelesen und in ein Dictio-
nary überführt. Anschließend wurde das Dictionary mit der Bibliothek pandas in ein
DataFrame konvertiert, um eine strukturierte Datenverarbeitung zu ermöglichen.
Da unterschiedliche Zutaten teils stark voneinander abweichende Wertebereiche in
ihren Geschmacksprofilen aufweisen können, wurden die Daten mittels Standardisierung normalisiert. Hierfür wurde der StandardScaler aus der Bibliothek scikit-learn
verwendet, um die Daten auf einen Mittelwert von null und eine Standardabweichung
von eins zu skalieren. Abschließend erfolgte die Aufteilung der normalisierten Daten
in Trainings- und Testsets im Verhältnis 80/20, um das Modell auf Generalisierung zu
prüfen und Overfitting vorzubeugen. Dies ermöglicht eine zuverlässige Bewertung der
Modellqualität und Robustheit gegenüber unbekannten Daten.

\subsection{Modellierung}
Die Entwicklung des Systems basiert auf einer Kombination aus Machine-Learning-Techniken und mathematischer Optimierung.

\subsection*{Datenbasis und Nutzerprofil}
Zunächst werden Zutaten mit Geschmacksprofilen und nutzerbezogene Fragen aus einer JSON-Datei geladen, um ein individuelles, gewichtetes Nutzerprofil zu erstellen. Der Nutzer durchläuft hierfür einen Fragebogen. Jede Antwort wird zu einem fünfdimensionalen Vektorprofil aggregiert:
\[
\mathbf{p}_{\text{user}} = \frac{1}{N} \sum_{i=1}^{N} \mathbf{q}_i,
\]
wobei \(\mathbf{q}_i\) das Profil der i-ten Frage ist und \(N\) die Gesamtanzahl beantworteter Fragen. Zusätzlich entscheidet der Nutzer, ob der Cocktail alkoholisch sein soll, und legt die maximale Zutatenzahl \(k\) fest.

\subsection*{Autoencoder zur Profilrekonstruktion}
Das zentrale Element des Systems ist ein Autoencoder, der das Nutzerprofil komprimiert und rekonstruiert. Der Autoencoder besteht aus:
\begin{itemize}
  \item \textbf{Encoder}: Komprimiert das Nutzerprofil auf einen dreidimensionalen Latent-Space:
    \[\mathbf{z} = \text{Encoder}(\mathbf{p}_{\text{user}})\]
  \item \textbf{Decoder}: Rekonstruiert das Profil aus dem Latent-Space:
    \[\mathbf{p}_{\text{rekon}} = \text{Decoder}(\mathbf{z})\]
\end{itemize}
Das Training erfolgt unüberwacht anhand der Geschmacksdaten mittels Loss-Funktion:
\[\mathcal{L} = \left\lVert \mathbf{p}_{\text{rekon}} - \mathbf{p}_{\text{user}} \right\rVert^2\]

\subsection*{Optimale Zutatenmischung}
Ausgehend von \( \mathbf{p}_{\text{rekon}} \) wird die optimale Mischung per Least-Squares berechnet:
\[\min_{\mathbf{w}} \ \left\lVert \mathbf{A}\mathbf{w} - \mathbf{p}_{\text{rekon}} \right\rVert^2\]
wobei:
\begin{itemize}
  \item \(\mathbf{A} \in \mathbb{R}^{5 \times M}\): Matrix der Zutatenprofile,
  \item \(\mathbf{w}\in \mathbb{R}^M\): Gewichtsvektor der Zutaten.
\end{itemize}
Negative Gewichte werden eliminiert, anschließend werden nur die Top-\(k\)-Zutaten verwendet und deren Anteile auf ein Gesamtvolumen von 200 ml normiert.

\chapter{Zusammenfassung und Ausblick}
Die in dieser Arbeit vorgestellte Implementierung zeigt eine modulare, gut struktu-
rierte Softwarelösung, die erfolgreich Methoden des maschinellen Lernens und der ma-
thematischen Optimierung kombiniert, um individuelle Cocktailrezepte automatisiert
zu generieren. Der entwickelte Ansatz zeichnet sich besonders durch seine Flexibili-
tät und Anpassungsfähigkeit aus, wodurch er leicht auf neue Zutaten oder geänderte
Anforderungen erweitert werden kann.
Ein entscheidender Vorteil des Systems ist seine Skalierbarkeit. Die modulare Architektur erlaubt es, weitere Komponenten, wie etwa neue Modelle zur Geschmacksanalyse
oder zusätzliche Optimierungsverfahren, mit geringem Aufwand zu integrieren. Trotz
dieser Stärken bestehen jedoch auch einige Herausforderungen. Vor allem die Qualität
und Vollständigkeit der zugrundeliegenden Datenbasis hat erheblichen Einfluss auf die
Präzision und Zuverlässigkeit der generierten Rezeptvorschläge. Fehlende oder schlecht
gewählte Zutatenprofile können zu weniger zufriedenstellenden Ergebnissen führen.
Um diesen Herausforderungen künftig besser zu begegnen, könnten mehrere Erweiterungen sinnvoll sein. Beispielsweise wäre die Implementierung einer benutzerfreundlichen, grafischen Benutzeroberfläche denkbar, die die Interaktion mit dem System
erheblich vereinfacht und den Nutzern die Möglichkeit bietet, unmittelbar Feedback
zu geben. Darüber hinaus könnte der Einsatz komplexerer Modelle wie Transformer-
Architekturen oder hybrider Ansätze, die traditionelle regelbasierte Systeme mit maschinellen Lernverfahren kombinieren, die Genauigkeit der Rezeptgenerierung weiter
steigern.
Zusätzlich könnte das Modell durch weitere Geschmacksdimensionen ergänzt werden, um die individuellen Präferenzen der Nutzer noch differenzierter abzubilden. Auch
eine Erweiterung um Nutzerfeedback und adaptive Lernmechanismen wie ReinforcementLearning könnte perspektivisch die Genauigkeit und Personalisierung des Systems
nachhaltig verbessern. Solche Ansätze würden nicht nur die Qualität der generierten
Rezepte verbessern, sondern auch das Potenzial des Systems im kommerziellen Einsatz,
etwa in Bars oder gastronomischen Betrieben, deutlich erhöhen.
\bibliographystyle{alpha}
\bibliography{literatur}


\chapter*{Selbstständigkeitserklärung}
Ich versichere, dass ich diese Arbeit selbstständig verfasst habe.

\vspace{2cm}
Stuttgart, \today \hfill (Florian Grassl)

\end{document}
