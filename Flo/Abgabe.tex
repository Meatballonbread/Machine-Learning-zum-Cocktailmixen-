\documentclass[a4paper,12pt]{article}
\usepackage[utf8]{inputenc}
\usepackage{geometry}
\geometry{a4paper, left=2.5cm, right=2.5cm, top=2.5cm, bottom=2.5cm}
\usepackage{setspace}
\onehalfspacing
\usepackage{graphicx}
\usepackage{amsmath}

\begin{document}

\title{Machine Learning zum Cocktailmixen}
\maketitle

\section{Programmierung}

Die Implementierung des Cocktailmixer-Systems erfolgte in Python unter Verwendung gängiger Bibliotheken für Machine Learning und Datenverarbeitung, insbesondere \texttt{pandas}, \texttt{numpy} und \texttt{scikit-learn}. Ziel war es, eine effiziente und skalierbare Software zu entwickeln, die Geschmacksprofile analysiert und daraus optimale Cocktailrezepte generiert.

\subsection{Softwarearchitektur}
Die Architektur des Programms folgt einem modularen Aufbau, um eine einfache Erweiterung und Wartung zu ermöglichen. Die Hauptkomponenten umfassen:
\begin{itemize}
    \item \textbf{Datenverarbeitung}: Ein Modul zur Verwaltung und Vorverarbeitung von Geschmacksdaten.
    \item \textbf{Maschinelles Lernen}: Implementierung eines Autoencoders zur Extraktion latenter Geschmackspräferenzen.
    \item \textbf{Optimierung}: Berechnung der optimalen Zutatenmischung mittels Least-Squares-Optimierung.
    \item \textbf{Benutzerschnittstelle}: Eine einfache API zur Interaktion mit dem System.
\end{itemize}

\subsection{Datenverwaltung}
Die Basis des Systems ist eine JSON-Datenbank, die Zutaten und ihre zugehörigen Geschmacksprofile speichert. Diese Daten werden mit \texttt{pandas} in ein DataFrame überführt und für das Training sowie die Optimierung genutzt. Ein Beispiel für die Datenstruktur:

\begin{verbatim}
{
    "Zutat": "Limettensaft",
    "Geschmack": [0.1, 0.9, 0.0, 0.2, 0.1]
}
\end{verbatim}

Die Normalisierung der Daten erfolgt mit \texttt{sklearn.preprocessing.StandardScaler}, um eine einheitliche Skalierung für das Machine-Learning-Modell zu gewährleisten.

\subsection{Modellimplementierung}
Das Herzstück des Systems ist ein Autoencoder, der in \texttt{TensorFlow/Keras} realisiert wurde. Die Architektur umfasst einen Encoder, der das Geschmacksprofil in eine niedrigere Dimension komprimiert, und einen Decoder, der es rekonstruiert:

\begin{verbatim}
from tensorflow.keras.models import Model
from tensorflow.keras.layers import Input, Dense

input_dim = 5
encoding_dim = 3

input_layer = Input(shape=(input_dim,))
encoded = Dense(encoding_dim, activation='relu')(input_layer)
decoded = Dense(input_dim, activation='sigmoid')(encoded)

autoencoder = Model(input_layer, decoded)
autoencoder.compile(optimizer='adam', loss='mse')
\end{verbatim}

Das Modell wird mit bestehenden Zutatendaten trainiert, um ein optimales Profiling der Nutzerpräferenzen zu ermöglichen.

\subsection{Optimierung der Zutatenmischung}
Nachdem das Nutzerprofil rekonstruiert wurde, erfolgt die Berechnung der optimalen Zutatenmischung mittels Least-Squares-Optimierung. Dies geschieht durch Minimierung der Differenz zwischen dem rekonstruierten Profil und den vorhandenen Zutatenkombinationen:

\begin{equation}
\min_{w} \| A w - p_{\text{rekon}} \|^2
\end{equation}

Hierbei ist \( A \) die Matrix der Zutatenprofile und \( w \) der Gewichtsvektor der Zutaten. Die Berechnung erfolgt mit \texttt{numpy.linalg.lstsq}.

\subsection{Zusammenfassung}
Die Implementierung des Cocktailmixers basiert auf einer strukturierten, modularen Architektur. Durch die Kombination von Machine Learning und mathematischer Optimierung wurde ein System entwickelt, das individuelle Cocktailrezepte generieren kann. Zukünftige Erweiterungen umfassen eine Benutzeroberfläche und die Integration zusätzlicher Zutatenprofile.

\end{document}
